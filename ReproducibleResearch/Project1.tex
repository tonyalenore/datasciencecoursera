% Options for packages loaded elsewhere
\PassOptionsToPackage{unicode}{hyperref}
\PassOptionsToPackage{hyphens}{url}
%
\documentclass[
]{article}
\usepackage{lmodern}
\usepackage{amsmath}
\usepackage{ifxetex,ifluatex}
\ifnum 0\ifxetex 1\fi\ifluatex 1\fi=0 % if pdftex
  \usepackage[T1]{fontenc}
  \usepackage[utf8]{inputenc}
  \usepackage{textcomp} % provide euro and other symbols
  \usepackage{amssymb}
\else % if luatex or xetex
  \usepackage{unicode-math}
  \defaultfontfeatures{Scale=MatchLowercase}
  \defaultfontfeatures[\rmfamily]{Ligatures=TeX,Scale=1}
\fi
% Use upquote if available, for straight quotes in verbatim environments
\IfFileExists{upquote.sty}{\usepackage{upquote}}{}
\IfFileExists{microtype.sty}{% use microtype if available
  \usepackage[]{microtype}
  \UseMicrotypeSet[protrusion]{basicmath} % disable protrusion for tt fonts
}{}
\makeatletter
\@ifundefined{KOMAClassName}{% if non-KOMA class
  \IfFileExists{parskip.sty}{%
    \usepackage{parskip}
  }{% else
    \setlength{\parindent}{0pt}
    \setlength{\parskip}{6pt plus 2pt minus 1pt}}
}{% if KOMA class
  \KOMAoptions{parskip=half}}
\makeatother
\usepackage{xcolor}
\IfFileExists{xurl.sty}{\usepackage{xurl}}{} % add URL line breaks if available
\IfFileExists{bookmark.sty}{\usepackage{bookmark}}{\usepackage{hyperref}}
\hypersetup{
  pdftitle={Reproducible Research Project 1},
  pdfauthor={Tonya MacDonald},
  hidelinks,
  pdfcreator={LaTeX via pandoc}}
\urlstyle{same} % disable monospaced font for URLs
\usepackage[margin=1in]{geometry}
\usepackage{color}
\usepackage{fancyvrb}
\newcommand{\VerbBar}{|}
\newcommand{\VERB}{\Verb[commandchars=\\\{\}]}
\DefineVerbatimEnvironment{Highlighting}{Verbatim}{commandchars=\\\{\}}
% Add ',fontsize=\small' for more characters per line
\usepackage{framed}
\definecolor{shadecolor}{RGB}{248,248,248}
\newenvironment{Shaded}{\begin{snugshade}}{\end{snugshade}}
\newcommand{\AlertTok}[1]{\textcolor[rgb]{0.94,0.16,0.16}{#1}}
\newcommand{\AnnotationTok}[1]{\textcolor[rgb]{0.56,0.35,0.01}{\textbf{\textit{#1}}}}
\newcommand{\AttributeTok}[1]{\textcolor[rgb]{0.77,0.63,0.00}{#1}}
\newcommand{\BaseNTok}[1]{\textcolor[rgb]{0.00,0.00,0.81}{#1}}
\newcommand{\BuiltInTok}[1]{#1}
\newcommand{\CharTok}[1]{\textcolor[rgb]{0.31,0.60,0.02}{#1}}
\newcommand{\CommentTok}[1]{\textcolor[rgb]{0.56,0.35,0.01}{\textit{#1}}}
\newcommand{\CommentVarTok}[1]{\textcolor[rgb]{0.56,0.35,0.01}{\textbf{\textit{#1}}}}
\newcommand{\ConstantTok}[1]{\textcolor[rgb]{0.00,0.00,0.00}{#1}}
\newcommand{\ControlFlowTok}[1]{\textcolor[rgb]{0.13,0.29,0.53}{\textbf{#1}}}
\newcommand{\DataTypeTok}[1]{\textcolor[rgb]{0.13,0.29,0.53}{#1}}
\newcommand{\DecValTok}[1]{\textcolor[rgb]{0.00,0.00,0.81}{#1}}
\newcommand{\DocumentationTok}[1]{\textcolor[rgb]{0.56,0.35,0.01}{\textbf{\textit{#1}}}}
\newcommand{\ErrorTok}[1]{\textcolor[rgb]{0.64,0.00,0.00}{\textbf{#1}}}
\newcommand{\ExtensionTok}[1]{#1}
\newcommand{\FloatTok}[1]{\textcolor[rgb]{0.00,0.00,0.81}{#1}}
\newcommand{\FunctionTok}[1]{\textcolor[rgb]{0.00,0.00,0.00}{#1}}
\newcommand{\ImportTok}[1]{#1}
\newcommand{\InformationTok}[1]{\textcolor[rgb]{0.56,0.35,0.01}{\textbf{\textit{#1}}}}
\newcommand{\KeywordTok}[1]{\textcolor[rgb]{0.13,0.29,0.53}{\textbf{#1}}}
\newcommand{\NormalTok}[1]{#1}
\newcommand{\OperatorTok}[1]{\textcolor[rgb]{0.81,0.36,0.00}{\textbf{#1}}}
\newcommand{\OtherTok}[1]{\textcolor[rgb]{0.56,0.35,0.01}{#1}}
\newcommand{\PreprocessorTok}[1]{\textcolor[rgb]{0.56,0.35,0.01}{\textit{#1}}}
\newcommand{\RegionMarkerTok}[1]{#1}
\newcommand{\SpecialCharTok}[1]{\textcolor[rgb]{0.00,0.00,0.00}{#1}}
\newcommand{\SpecialStringTok}[1]{\textcolor[rgb]{0.31,0.60,0.02}{#1}}
\newcommand{\StringTok}[1]{\textcolor[rgb]{0.31,0.60,0.02}{#1}}
\newcommand{\VariableTok}[1]{\textcolor[rgb]{0.00,0.00,0.00}{#1}}
\newcommand{\VerbatimStringTok}[1]{\textcolor[rgb]{0.31,0.60,0.02}{#1}}
\newcommand{\WarningTok}[1]{\textcolor[rgb]{0.56,0.35,0.01}{\textbf{\textit{#1}}}}
\usepackage{graphicx}
\makeatletter
\def\maxwidth{\ifdim\Gin@nat@width>\linewidth\linewidth\else\Gin@nat@width\fi}
\def\maxheight{\ifdim\Gin@nat@height>\textheight\textheight\else\Gin@nat@height\fi}
\makeatother
% Scale images if necessary, so that they will not overflow the page
% margins by default, and it is still possible to overwrite the defaults
% using explicit options in \includegraphics[width, height, ...]{}
\setkeys{Gin}{width=\maxwidth,height=\maxheight,keepaspectratio}
% Set default figure placement to htbp
\makeatletter
\def\fps@figure{htbp}
\makeatother
\setlength{\emergencystretch}{3em} % prevent overfull lines
\providecommand{\tightlist}{%
  \setlength{\itemsep}{0pt}\setlength{\parskip}{0pt}}
\setcounter{secnumdepth}{-\maxdimen} % remove section numbering
\ifluatex
  \usepackage{selnolig}  % disable illegal ligatures
\fi

\title{Reproducible Research Project 1}
\author{Tonya MacDonald}
\date{2/15/2021}

\begin{document}
\maketitle

\hypertarget{load-data}{%
\subsubsection{Load Data}\label{load-data}}

Download and unzip data, then load into data table

\begin{Shaded}
\begin{Highlighting}[]
\FunctionTok{library}\NormalTok{(}\StringTok{"data.table"}\NormalTok{)}
\FunctionTok{library}\NormalTok{(ggplot2)}

\NormalTok{fileUrl }\OtherTok{\textless{}{-}} \StringTok{"https://d396qusza40orc.cloudfront.net/repdata\%2Fdata\%2Factivity.zip"}

\FunctionTok{download.file}\NormalTok{(fileUrl, }\AttributeTok{destfile =} \FunctionTok{paste0}\NormalTok{(}\FunctionTok{getwd}\NormalTok{(), }\StringTok{\textquotesingle{}/repdata\%2Fdata\%2Factivity.zip\textquotesingle{}}\NormalTok{), }\AttributeTok{method =} \StringTok{"curl"}\NormalTok{)}

\FunctionTok{unzip}\NormalTok{(}\StringTok{"repdata\%2Fdata\%2Factivity.zip"}\NormalTok{,}\AttributeTok{exdir =} \StringTok{"data"}\NormalTok{)}

\NormalTok{activitydata }\OtherTok{\textless{}{-}}\NormalTok{ data.table}\SpecialCharTok{::}\FunctionTok{fread}\NormalTok{(}\AttributeTok{input =} \StringTok{"data/activity.csv"}\NormalTok{)}
\end{Highlighting}
\end{Shaded}

\hypertarget{what-is-mean-total-number-of-steps-taken-per-day}{%
\subsection{What is mean total number of steps taken per
day?}\label{what-is-mean-total-number-of-steps-taken-per-day}}

\begin{enumerate}
\def\labelenumi{\arabic{enumi}.}
\tightlist
\item
  Calculate the total number of steps taken per day
\end{enumerate}

\begin{Shaded}
\begin{Highlighting}[]
\CommentTok{\# sum the total steps}
\NormalTok{totalsteps }\OtherTok{\textless{}{-}} \FunctionTok{as.data.table}\NormalTok{(}\FunctionTok{setNames}\NormalTok{(}\FunctionTok{aggregate}\NormalTok{(activitydata}\SpecialCharTok{$}\NormalTok{steps, }\AttributeTok{by=}\FunctionTok{list}\NormalTok{(activitydata}\SpecialCharTok{$}\NormalTok{date), }\AttributeTok{FUN=}\NormalTok{sum), }\FunctionTok{c}\NormalTok{(}\StringTok{"date"}\NormalTok{,}\StringTok{"steps"}\NormalTok{)))}
\end{Highlighting}
\end{Shaded}

\begin{Shaded}
\begin{Highlighting}[]
\CommentTok{\# take a look at the first 10 rows}
\FunctionTok{head}\NormalTok{(totalsteps, }\DecValTok{10}\NormalTok{)}
\end{Highlighting}
\end{Shaded}

\begin{verbatim}
##           date steps
##  1: 2012-10-01    NA
##  2: 2012-10-02   126
##  3: 2012-10-03 11352
##  4: 2012-10-04 12116
##  5: 2012-10-05 13294
##  6: 2012-10-06 15420
##  7: 2012-10-07 11015
##  8: 2012-10-08    NA
##  9: 2012-10-09 12811
## 10: 2012-10-10  9900
\end{verbatim}

\begin{enumerate}
\def\labelenumi{\arabic{enumi}.}
\setcounter{enumi}{1}
\tightlist
\item
  Make a histogram of the total number of steps per day
\end{enumerate}

\begin{Shaded}
\begin{Highlighting}[]
\CommentTok{\# histogram}
\FunctionTok{ggplot}\NormalTok{(totalsteps, }\FunctionTok{aes}\NormalTok{(}\AttributeTok{x =}\NormalTok{ steps)) }\SpecialCharTok{+}
    \FunctionTok{geom\_histogram}\NormalTok{(}\AttributeTok{fill =} \StringTok{"pink"}\NormalTok{, }\AttributeTok{binwidth =} \DecValTok{1000}\NormalTok{) }\SpecialCharTok{+}
    \FunctionTok{labs}\NormalTok{(}\AttributeTok{title =} \StringTok{"Steps per Day"}\NormalTok{, }\AttributeTok{x =} \StringTok{"Steps"}\NormalTok{, }\AttributeTok{y =} \StringTok{"Frequency"}\NormalTok{) }\SpecialCharTok{+} 
    \FunctionTok{theme\_bw}\NormalTok{()}
\end{Highlighting}
\end{Shaded}

\begin{verbatim}
## Warning: Removed 8 rows containing non-finite values (stat_bin).
\end{verbatim}

\includegraphics{Project1_files/figure-latex/unnamed-chunk-4-1.pdf}

\begin{enumerate}
\def\labelenumi{\arabic{enumi}.}
\setcounter{enumi}{2}
\tightlist
\item
  Calculate the mean and median of the total number of steps taken per
  day
\end{enumerate}

\begin{Shaded}
\begin{Highlighting}[]
\CommentTok{\# average steps}
\NormalTok{meansteps }\OtherTok{\textless{}{-}} \FunctionTok{mean}\NormalTok{(totalsteps}\SpecialCharTok{$}\NormalTok{steps, }\AttributeTok{na.rm=}\ConstantTok{TRUE}\NormalTok{)}
\NormalTok{meansteps}
\end{Highlighting}
\end{Shaded}

\begin{verbatim}
## [1] 10766.19
\end{verbatim}

\begin{Shaded}
\begin{Highlighting}[]
\CommentTok{\# median step value}
\NormalTok{mediansteps }\OtherTok{\textless{}{-}} \FunctionTok{median}\NormalTok{(totalsteps}\SpecialCharTok{$}\NormalTok{steps, }\AttributeTok{na.rm=}\ConstantTok{TRUE}\NormalTok{)}
\NormalTok{mediansteps}
\end{Highlighting}
\end{Shaded}

\begin{verbatim}
## [1] 10765
\end{verbatim}

\hypertarget{what-is-the-average-daily-activity-pattern}{%
\subsection{What is the average daily activity
pattern?}\label{what-is-the-average-daily-activity-pattern}}

\begin{enumerate}
\def\labelenumi{\arabic{enumi}.}
\tightlist
\item
  Make a time series plot (i.e.~𝚝𝚢𝚙𝚎 = ``𝚕'') of the 5-minute interval
  (x-axis) and the average number of steps taken, averaged across all
  days (y-axis)
\end{enumerate}

\begin{Shaded}
\begin{Highlighting}[]
\CommentTok{\# find all the intervals}
\NormalTok{intervals }\OtherTok{\textless{}{-}} \FunctionTok{as.data.table}\NormalTok{(}\FunctionTok{setNames}\NormalTok{(}\FunctionTok{aggregate}\NormalTok{(activitydata}\SpecialCharTok{$}\NormalTok{steps, }\AttributeTok{by=}\FunctionTok{list}\NormalTok{(activitydata}\SpecialCharTok{$}\NormalTok{interval), }\AttributeTok{FUN=}\NormalTok{sum, }\AttributeTok{na.rm=}\ConstantTok{TRUE}\NormalTok{), }\FunctionTok{c}\NormalTok{(}\StringTok{"interval"}\NormalTok{,}\StringTok{"steps"}\NormalTok{)))}

\CommentTok{\# show the data}
\FunctionTok{head}\NormalTok{(intervals,}\DecValTok{10}\NormalTok{)}
\end{Highlighting}
\end{Shaded}

\begin{verbatim}
##     interval steps
##  1:        0    91
##  2:        5    18
##  3:       10     7
##  4:       15     8
##  5:       20     4
##  6:       25   111
##  7:       30    28
##  8:       35    46
##  9:       40     0
## 10:       45    78
\end{verbatim}

\begin{Shaded}
\begin{Highlighting}[]
\CommentTok{\# line chart}
\FunctionTok{ggplot}\NormalTok{(intervals, }\FunctionTok{aes}\NormalTok{(}\AttributeTok{x =}\NormalTok{ interval , }\AttributeTok{y =}\NormalTok{ steps)) }\SpecialCharTok{+} 
  \FunctionTok{geom\_line}\NormalTok{(}\AttributeTok{color=}\StringTok{"pink"}\NormalTok{, }\AttributeTok{size=}\DecValTok{1}\NormalTok{) }\SpecialCharTok{+} 
  \FunctionTok{labs}\NormalTok{(}\AttributeTok{title =} \StringTok{"Average Steps per Day"}\NormalTok{, }\AttributeTok{x =} \StringTok{"Interval"}\NormalTok{, }\AttributeTok{y =} \StringTok{"Average Steps per Day"}\NormalTok{) }\SpecialCharTok{+} 
  \FunctionTok{theme\_bw}\NormalTok{()}
\end{Highlighting}
\end{Shaded}

\includegraphics{Project1_files/figure-latex/unnamed-chunk-7-1.pdf}

\begin{enumerate}
\def\labelenumi{\arabic{enumi}.}
\setcounter{enumi}{1}
\tightlist
\item
  Which 5-minute interval, on average across all the days in the
  dataset, contains the maximum number of steps?
\end{enumerate}

\begin{Shaded}
\begin{Highlighting}[]
\CommentTok{\# find the interval with the max steps}
\NormalTok{intervals[steps }\SpecialCharTok{==} \FunctionTok{max}\NormalTok{(steps)]}\SpecialCharTok{$}\NormalTok{interval}
\end{Highlighting}
\end{Shaded}

\begin{verbatim}
## [1] 835
\end{verbatim}

\hypertarget{imputing-missing-values}{%
\subsection{Imputing missing values}\label{imputing-missing-values}}

\begin{enumerate}
\def\labelenumi{\arabic{enumi}.}
\tightlist
\item
  Calculate and report the total number of missing values in the data
\end{enumerate}

\begin{Shaded}
\begin{Highlighting}[]
\CommentTok{\#number of rows with NAs}
\FunctionTok{nrow}\NormalTok{(activitydata[}\FunctionTok{is.na}\NormalTok{(steps),])}
\end{Highlighting}
\end{Shaded}

\begin{verbatim}
## [1] 2304
\end{verbatim}

\begin{enumerate}
\def\labelenumi{\arabic{enumi}.}
\setcounter{enumi}{1}
\tightlist
\item
  \& 3. Devise a strategy for filling in all of the missing values in
  the dataset. The strategy does not need to be sophisticated. For
  example, you could use the mean/median for that day, or the mean for
  that 5-minute interval, etc. Create a new dataset that is equal to the
  original dataset but with the missing data filled in.
\end{enumerate}

\begin{Shaded}
\begin{Highlighting}[]
\CommentTok{\# replace NAs with the medin}
\NormalTok{activitydata[}\FunctionTok{is.na}\NormalTok{(steps), }\StringTok{"steps"}\NormalTok{] }\OtherTok{\textless{}{-}}\NormalTok{ activitydata[, }\FunctionTok{c}\NormalTok{(}\FunctionTok{lapply}\NormalTok{(.SD, median, }\AttributeTok{na.rm =} \ConstantTok{TRUE}\NormalTok{)), .SDcols }\OtherTok{=} \FunctionTok{c}\NormalTok{(}\StringTok{"steps"}\NormalTok{)]}

\CommentTok{\# verify no more NAs}
\FunctionTok{nrow}\NormalTok{(activitydata[}\FunctionTok{is.na}\NormalTok{(steps),])}
\end{Highlighting}
\end{Shaded}

\begin{verbatim}
## [1] 0
\end{verbatim}

\begin{enumerate}
\def\labelenumi{\arabic{enumi}.}
\setcounter{enumi}{3}
\tightlist
\item
  Make a histogram of the total number of steps taken each day and
  calculate and report the mean and median total number of steps taken
  per day. Do these values differ from the estimates from the first part
  of the assignment? What is the impact of imputing missing data on the
  estimates of the total daily number of steps?
\end{enumerate}

\begin{Shaded}
\begin{Highlighting}[]
\CommentTok{\# sum total steps again, this time with the NAs replaced with the median}
\NormalTok{totalsteps2 }\OtherTok{\textless{}{-}} \FunctionTok{as.data.table}\NormalTok{(}\FunctionTok{setNames}\NormalTok{(}\FunctionTok{aggregate}\NormalTok{(activitydata}\SpecialCharTok{$}\NormalTok{steps, }\AttributeTok{by=}\FunctionTok{list}\NormalTok{(activitydata}\SpecialCharTok{$}\NormalTok{date), }\AttributeTok{FUN=}\NormalTok{sum), }\FunctionTok{c}\NormalTok{(}\StringTok{"date"}\NormalTok{,}\StringTok{"steps"}\NormalTok{)))}

\CommentTok{\# view data}
\FunctionTok{head}\NormalTok{(totalsteps2)}
\end{Highlighting}
\end{Shaded}

\begin{verbatim}
##          date steps
## 1: 2012-10-01     0
## 2: 2012-10-02   126
## 3: 2012-10-03 11352
## 4: 2012-10-04 12116
## 5: 2012-10-05 13294
## 6: 2012-10-06 15420
\end{verbatim}

\begin{Shaded}
\begin{Highlighting}[]
\CommentTok{\# average the total steps with NAs removed}
\NormalTok{meansteps2 }\OtherTok{\textless{}{-}} \FunctionTok{mean}\NormalTok{(totalsteps2}\SpecialCharTok{$}\NormalTok{steps)}
\NormalTok{meansteps2}
\end{Highlighting}
\end{Shaded}

\begin{verbatim}
## [1] 9354.23
\end{verbatim}

\begin{Shaded}
\begin{Highlighting}[]
\CommentTok{\# median should still be the same}
\NormalTok{mediansteps2 }\OtherTok{\textless{}{-}} \FunctionTok{median}\NormalTok{(totalsteps2}\SpecialCharTok{$}\NormalTok{steps)}
\NormalTok{mediansteps2}
\end{Highlighting}
\end{Shaded}

\begin{verbatim}
## [1] 10395
\end{verbatim}

\begin{Shaded}
\begin{Highlighting}[]
\CommentTok{\# histogram}
\FunctionTok{ggplot}\NormalTok{(totalsteps, }\FunctionTok{aes}\NormalTok{(}\AttributeTok{x =}\NormalTok{ steps)) }\SpecialCharTok{+}
    \FunctionTok{geom\_histogram}\NormalTok{(}\AttributeTok{fill =} \StringTok{"pink"}\NormalTok{, }\AttributeTok{binwidth =} \DecValTok{1000}\NormalTok{) }\SpecialCharTok{+}
    \FunctionTok{labs}\NormalTok{(}\AttributeTok{title =} \StringTok{"Steps per Day"}\NormalTok{, }\AttributeTok{x =} \StringTok{"Steps"}\NormalTok{, }\AttributeTok{y =} \StringTok{"Frequency"}\NormalTok{) }\SpecialCharTok{+} 
    \FunctionTok{theme\_bw}\NormalTok{()}
\end{Highlighting}
\end{Shaded}

\begin{verbatim}
## Warning: Removed 8 rows containing non-finite values (stat_bin).
\end{verbatim}

\includegraphics{Project1_files/figure-latex/unnamed-chunk-11-1.pdf}

Compare the mean and median before and after NA removal

\begin{Shaded}
\begin{Highlighting}[]
\NormalTok{meansteps}
\end{Highlighting}
\end{Shaded}

\begin{verbatim}
## [1] 10766.19
\end{verbatim}

\begin{Shaded}
\begin{Highlighting}[]
\NormalTok{meansteps2}
\end{Highlighting}
\end{Shaded}

\begin{verbatim}
## [1] 9354.23
\end{verbatim}

\begin{Shaded}
\begin{Highlighting}[]
\NormalTok{mediansteps}
\end{Highlighting}
\end{Shaded}

\begin{verbatim}
## [1] 10765
\end{verbatim}

\begin{Shaded}
\begin{Highlighting}[]
\NormalTok{mediansteps2}
\end{Highlighting}
\end{Shaded}

\begin{verbatim}
## [1] 10395
\end{verbatim}

\hypertarget{are-there-differences-in-activity-patterns-between-weekdays-and-weekends}{%
\subsection{Are there differences in activity patterns between weekdays
and
weekends?}\label{are-there-differences-in-activity-patterns-between-weekdays-and-weekends}}

\begin{enumerate}
\def\labelenumi{\arabic{enumi}.}
\tightlist
\item
  Create a new factor variable in the dataset with two levels --
  ``weekday'' and ``weekend'' indicating whether a given date is a
  weekday or weekend day.
\end{enumerate}

\begin{Shaded}
\begin{Highlighting}[]
\CommentTok{\#classify dates as weekdays or weekends}
\NormalTok{activitydata[, }\StringTok{\textasciigrave{}}\AttributeTok{Day of Week}\StringTok{\textasciigrave{}}\SpecialCharTok{:}\ErrorTok{=} \FunctionTok{weekdays}\NormalTok{(}\AttributeTok{x =}\NormalTok{ date)]}
\NormalTok{activitydata[}\FunctionTok{grepl}\NormalTok{(}\AttributeTok{pattern =} \StringTok{"Monday|Tuesday|Wednesday|Thursday|Friday"}\NormalTok{, }\AttributeTok{x =} \StringTok{\textasciigrave{}}\AttributeTok{Day of Week}\StringTok{\textasciigrave{}}\NormalTok{), }\StringTok{"weekday or weekend"}\NormalTok{] }\OtherTok{\textless{}{-}} \StringTok{"weekday"}
\NormalTok{activitydata[}\FunctionTok{grepl}\NormalTok{(}\AttributeTok{pattern =} \StringTok{"Saturday|Sunday"}\NormalTok{, }\AttributeTok{x =} \StringTok{\textasciigrave{}}\AttributeTok{Day of Week}\StringTok{\textasciigrave{}}\NormalTok{), }\StringTok{"weekday or weekend"}\NormalTok{] }\OtherTok{\textless{}{-}} \StringTok{"weekend"}
\NormalTok{activitydata[, }\StringTok{\textasciigrave{}}\AttributeTok{weekday or weekend}\StringTok{\textasciigrave{}} \SpecialCharTok{:}\ErrorTok{=} \FunctionTok{as.factor}\NormalTok{(}\StringTok{\textasciigrave{}}\AttributeTok{weekday or weekend}\StringTok{\textasciigrave{}}\NormalTok{)]}
\FunctionTok{head}\NormalTok{(activitydata, }\DecValTok{10}\NormalTok{)}
\end{Highlighting}
\end{Shaded}

\begin{verbatim}
##     steps       date interval Day of Week weekday or weekend
##  1:     0 2012-10-01        0      Monday            weekday
##  2:     0 2012-10-01        5      Monday            weekday
##  3:     0 2012-10-01       10      Monday            weekday
##  4:     0 2012-10-01       15      Monday            weekday
##  5:     0 2012-10-01       20      Monday            weekday
##  6:     0 2012-10-01       25      Monday            weekday
##  7:     0 2012-10-01       30      Monday            weekday
##  8:     0 2012-10-01       35      Monday            weekday
##  9:     0 2012-10-01       40      Monday            weekday
## 10:     0 2012-10-01       45      Monday            weekday
\end{verbatim}

\begin{enumerate}
\def\labelenumi{\arabic{enumi}.}
\setcounter{enumi}{1}
\tightlist
\item
  Make a panel plot containing a time series plot (i.e.~𝚝𝚢𝚙𝚎 = ``𝚕'') of
  the 5-minute interval (x-axis) and the average number of steps taken,
  averaged across all weekday days or weekend days (y-axis). See the
  README file in the GitHub repository to see an example of what this
  plot should look like using simulated data.
\end{enumerate}

\begin{Shaded}
\begin{Highlighting}[]
\NormalTok{activitydata[}\FunctionTok{is.na}\NormalTok{(steps), }\StringTok{"steps"}\NormalTok{] }\OtherTok{\textless{}{-}}\NormalTok{ activitydata[, }\FunctionTok{c}\NormalTok{(}\FunctionTok{lapply}\NormalTok{(.SD, median, }\AttributeTok{na.rm =} \ConstantTok{TRUE}\NormalTok{)), .SDcols }\OtherTok{=} \FunctionTok{c}\NormalTok{(}\StringTok{"steps"}\NormalTok{)]}
\NormalTok{IntervalDT }\OtherTok{\textless{}{-}}\NormalTok{ activitydata[, }\FunctionTok{c}\NormalTok{(}\FunctionTok{lapply}\NormalTok{(.SD, mean, }\AttributeTok{na.rm =} \ConstantTok{TRUE}\NormalTok{)), .SDcols }\OtherTok{=} \FunctionTok{c}\NormalTok{(}\StringTok{"steps"}\NormalTok{), by }\OtherTok{=}\NormalTok{ .(interval, }\StringTok{\textasciigrave{}}\AttributeTok{weekday or weekend}\StringTok{\textasciigrave{}}\NormalTok{)]}

\FunctionTok{ggplot}\NormalTok{(IntervalDT , }\FunctionTok{aes}\NormalTok{(}\AttributeTok{x =}\NormalTok{ interval , }\AttributeTok{y =}\NormalTok{ steps, }\AttributeTok{color=}\StringTok{\textasciigrave{}}\AttributeTok{weekday or weekend}\StringTok{\textasciigrave{}}\NormalTok{)) }\SpecialCharTok{+} \FunctionTok{geom\_line}\NormalTok{() }\SpecialCharTok{+} \FunctionTok{labs}\NormalTok{(}\AttributeTok{title =} \StringTok{"Average Steps per Day"}\NormalTok{, }\AttributeTok{x =} \StringTok{"Interval"}\NormalTok{, }\AttributeTok{y =} \StringTok{"Steps"}\NormalTok{) }\SpecialCharTok{+} \FunctionTok{facet\_wrap}\NormalTok{(}\SpecialCharTok{\textasciitilde{}}\StringTok{\textasciigrave{}}\AttributeTok{weekday or weekend}\StringTok{\textasciigrave{}}\NormalTok{ , }\AttributeTok{ncol =} \DecValTok{1}\NormalTok{, }\AttributeTok{nrow=}\DecValTok{2}\NormalTok{)}
\end{Highlighting}
\end{Shaded}

\includegraphics{Project1_files/figure-latex/unnamed-chunk-14-1.pdf}

\end{document}
