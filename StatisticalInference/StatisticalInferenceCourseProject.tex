% Options for packages loaded elsewhere
\PassOptionsToPackage{unicode}{hyperref}
\PassOptionsToPackage{hyphens}{url}
%
\documentclass[
]{article}
\usepackage{lmodern}
\usepackage{amsmath}
\usepackage{ifxetex,ifluatex}
\ifnum 0\ifxetex 1\fi\ifluatex 1\fi=0 % if pdftex
  \usepackage[T1]{fontenc}
  \usepackage[utf8]{inputenc}
  \usepackage{textcomp} % provide euro and other symbols
  \usepackage{amssymb}
\else % if luatex or xetex
  \usepackage{unicode-math}
  \defaultfontfeatures{Scale=MatchLowercase}
  \defaultfontfeatures[\rmfamily]{Ligatures=TeX,Scale=1}
\fi
% Use upquote if available, for straight quotes in verbatim environments
\IfFileExists{upquote.sty}{\usepackage{upquote}}{}
\IfFileExists{microtype.sty}{% use microtype if available
  \usepackage[]{microtype}
  \UseMicrotypeSet[protrusion]{basicmath} % disable protrusion for tt fonts
}{}
\makeatletter
\@ifundefined{KOMAClassName}{% if non-KOMA class
  \IfFileExists{parskip.sty}{%
    \usepackage{parskip}
  }{% else
    \setlength{\parindent}{0pt}
    \setlength{\parskip}{6pt plus 2pt minus 1pt}}
}{% if KOMA class
  \KOMAoptions{parskip=half}}
\makeatother
\usepackage{xcolor}
\IfFileExists{xurl.sty}{\usepackage{xurl}}{} % add URL line breaks if available
\IfFileExists{bookmark.sty}{\usepackage{bookmark}}{\usepackage{hyperref}}
\hypersetup{
  pdftitle={Statistical Inference Course Project},
  pdfauthor={Tonya MacDonald},
  hidelinks,
  pdfcreator={LaTeX via pandoc}}
\urlstyle{same} % disable monospaced font for URLs
\usepackage[margin=1in]{geometry}
\usepackage{color}
\usepackage{fancyvrb}
\newcommand{\VerbBar}{|}
\newcommand{\VERB}{\Verb[commandchars=\\\{\}]}
\DefineVerbatimEnvironment{Highlighting}{Verbatim}{commandchars=\\\{\}}
% Add ',fontsize=\small' for more characters per line
\usepackage{framed}
\definecolor{shadecolor}{RGB}{248,248,248}
\newenvironment{Shaded}{\begin{snugshade}}{\end{snugshade}}
\newcommand{\AlertTok}[1]{\textcolor[rgb]{0.94,0.16,0.16}{#1}}
\newcommand{\AnnotationTok}[1]{\textcolor[rgb]{0.56,0.35,0.01}{\textbf{\textit{#1}}}}
\newcommand{\AttributeTok}[1]{\textcolor[rgb]{0.77,0.63,0.00}{#1}}
\newcommand{\BaseNTok}[1]{\textcolor[rgb]{0.00,0.00,0.81}{#1}}
\newcommand{\BuiltInTok}[1]{#1}
\newcommand{\CharTok}[1]{\textcolor[rgb]{0.31,0.60,0.02}{#1}}
\newcommand{\CommentTok}[1]{\textcolor[rgb]{0.56,0.35,0.01}{\textit{#1}}}
\newcommand{\CommentVarTok}[1]{\textcolor[rgb]{0.56,0.35,0.01}{\textbf{\textit{#1}}}}
\newcommand{\ConstantTok}[1]{\textcolor[rgb]{0.00,0.00,0.00}{#1}}
\newcommand{\ControlFlowTok}[1]{\textcolor[rgb]{0.13,0.29,0.53}{\textbf{#1}}}
\newcommand{\DataTypeTok}[1]{\textcolor[rgb]{0.13,0.29,0.53}{#1}}
\newcommand{\DecValTok}[1]{\textcolor[rgb]{0.00,0.00,0.81}{#1}}
\newcommand{\DocumentationTok}[1]{\textcolor[rgb]{0.56,0.35,0.01}{\textbf{\textit{#1}}}}
\newcommand{\ErrorTok}[1]{\textcolor[rgb]{0.64,0.00,0.00}{\textbf{#1}}}
\newcommand{\ExtensionTok}[1]{#1}
\newcommand{\FloatTok}[1]{\textcolor[rgb]{0.00,0.00,0.81}{#1}}
\newcommand{\FunctionTok}[1]{\textcolor[rgb]{0.00,0.00,0.00}{#1}}
\newcommand{\ImportTok}[1]{#1}
\newcommand{\InformationTok}[1]{\textcolor[rgb]{0.56,0.35,0.01}{\textbf{\textit{#1}}}}
\newcommand{\KeywordTok}[1]{\textcolor[rgb]{0.13,0.29,0.53}{\textbf{#1}}}
\newcommand{\NormalTok}[1]{#1}
\newcommand{\OperatorTok}[1]{\textcolor[rgb]{0.81,0.36,0.00}{\textbf{#1}}}
\newcommand{\OtherTok}[1]{\textcolor[rgb]{0.56,0.35,0.01}{#1}}
\newcommand{\PreprocessorTok}[1]{\textcolor[rgb]{0.56,0.35,0.01}{\textit{#1}}}
\newcommand{\RegionMarkerTok}[1]{#1}
\newcommand{\SpecialCharTok}[1]{\textcolor[rgb]{0.00,0.00,0.00}{#1}}
\newcommand{\SpecialStringTok}[1]{\textcolor[rgb]{0.31,0.60,0.02}{#1}}
\newcommand{\StringTok}[1]{\textcolor[rgb]{0.31,0.60,0.02}{#1}}
\newcommand{\VariableTok}[1]{\textcolor[rgb]{0.00,0.00,0.00}{#1}}
\newcommand{\VerbatimStringTok}[1]{\textcolor[rgb]{0.31,0.60,0.02}{#1}}
\newcommand{\WarningTok}[1]{\textcolor[rgb]{0.56,0.35,0.01}{\textbf{\textit{#1}}}}
\usepackage{graphicx}
\makeatletter
\def\maxwidth{\ifdim\Gin@nat@width>\linewidth\linewidth\else\Gin@nat@width\fi}
\def\maxheight{\ifdim\Gin@nat@height>\textheight\textheight\else\Gin@nat@height\fi}
\makeatother
% Scale images if necessary, so that they will not overflow the page
% margins by default, and it is still possible to overwrite the defaults
% using explicit options in \includegraphics[width, height, ...]{}
\setkeys{Gin}{width=\maxwidth,height=\maxheight,keepaspectratio}
% Set default figure placement to htbp
\makeatletter
\def\fps@figure{htbp}
\makeatother
\setlength{\emergencystretch}{3em} % prevent overfull lines
\providecommand{\tightlist}{%
  \setlength{\itemsep}{0pt}\setlength{\parskip}{0pt}}
\setcounter{secnumdepth}{-\maxdimen} % remove section numbering
\ifluatex
  \usepackage{selnolig}  % disable illegal ligatures
\fi

\title{Statistical Inference Course Project}
\author{Tonya MacDonald}
\date{}

\begin{document}
\maketitle

\hypertarget{purpose}{%
\subsubsection{Purpose}\label{purpose}}

Investigate the exponential distribution in R and compare it with the
Central Limit Theorem.

\hypertarget{libraries}{%
\subsubsection{Libraries}\label{libraries}}

\begin{Shaded}
\begin{Highlighting}[]
\FunctionTok{library}\NormalTok{(}\StringTok{"data.table"}\NormalTok{)}
\FunctionTok{library}\NormalTok{(}\StringTok{"ggplot2"}\NormalTok{)}
\end{Highlighting}
\end{Shaded}

\hypertarget{setup}{%
\subsubsection{Setup}\label{setup}}

\begin{Shaded}
\begin{Highlighting}[]
\FunctionTok{set.seed}\NormalTok{(}\DecValTok{1983}\NormalTok{)}
\CommentTok{\# set lambda to 0.2, sample size to 40, 1000 simulations}
\NormalTok{l }\OtherTok{\textless{}{-}} \FloatTok{0.2}
\NormalTok{n }\OtherTok{\textless{}{-}} \DecValTok{40}
\NormalTok{s }\OtherTok{\textless{}{-}} \DecValTok{1000}

\CommentTok{\# calculate means}
\NormalTok{means }\OtherTok{\textless{}{-}} \FunctionTok{apply}\NormalTok{(}\FunctionTok{replicate}\NormalTok{(s, }\FunctionTok{rexp}\NormalTok{(n, l)), }\DecValTok{2}\NormalTok{, mean)}
\end{Highlighting}
\end{Shaded}

\hypertarget{sample-mean-versus-theoretical-mean}{%
\subsubsection{1 - Sample Mean versus Theoretical
Mean}\label{sample-mean-versus-theoretical-mean}}

Show the sample mean and compare it to the theoretical mean of the
distribution.

\begin{Shaded}
\begin{Highlighting}[]
\CommentTok{\# sample mean}
\NormalTok{sample\_mean }\OtherTok{\textless{}{-}} \FunctionTok{mean}\NormalTok{(means)}
\NormalTok{sample\_mean}
\end{Highlighting}
\end{Shaded}

\begin{verbatim}
## [1] 5.027677
\end{verbatim}

\begin{Shaded}
\begin{Highlighting}[]
\CommentTok{\# theoretical mean}
\NormalTok{theory\_mean }\OtherTok{\textless{}{-}} \DecValTok{1}\SpecialCharTok{/}\NormalTok{l}
\NormalTok{theory\_mean}
\end{Highlighting}
\end{Shaded}

\begin{verbatim}
## [1] 5
\end{verbatim}

\begin{Shaded}
\begin{Highlighting}[]
\CommentTok{\# histogram}
\FunctionTok{hist}\NormalTok{(means, }\AttributeTok{xlab =} \StringTok{"mean"}\NormalTok{, }\AttributeTok{main=}\StringTok{"Sample Mean versus Population Mean"}\NormalTok{)}
\FunctionTok{abline}\NormalTok{(}\AttributeTok{v =}\NormalTok{ sample\_mean, }\AttributeTok{col =} \StringTok{"red"}\NormalTok{, )}
\FunctionTok{abline}\NormalTok{(}\AttributeTok{v =}\NormalTok{ theory\_mean, }\AttributeTok{col =} \StringTok{"blue"}\NormalTok{)}
\end{Highlighting}
\end{Shaded}

\includegraphics{StatisticalInferenceCourseProject_files/figure-latex/unnamed-chunk-3-1.pdf}

\hypertarget{sample-variance-versus-theoretical-variance}{%
\subsubsection{2 - Sample Variance versus Theoretical
Variance}\label{sample-variance-versus-theoretical-variance}}

Show how variable the sample is (via variance) and compare it to the
theoretical variance of the distribution.

\begin{Shaded}
\begin{Highlighting}[]
\CommentTok{\# standard deviation of distribution}
\NormalTok{sd\_dist }\OtherTok{\textless{}{-}} \FunctionTok{sd}\NormalTok{(means)}
\NormalTok{sd\_dist}
\end{Highlighting}
\end{Shaded}

\begin{verbatim}
## [1] 0.7857763
\end{verbatim}

\begin{Shaded}
\begin{Highlighting}[]
\CommentTok{\# standard deviation }
\NormalTok{sd\_theory }\OtherTok{\textless{}{-}}\NormalTok{ (}\DecValTok{1}\SpecialCharTok{/}\NormalTok{l)}\SpecialCharTok{/}\FunctionTok{sqrt}\NormalTok{(n)}
\NormalTok{sd\_theory}
\end{Highlighting}
\end{Shaded}

\begin{verbatim}
## [1] 0.7905694
\end{verbatim}

\begin{Shaded}
\begin{Highlighting}[]
\CommentTok{\# variance of distribution}
\NormalTok{v\_dist }\OtherTok{\textless{}{-}}\NormalTok{ sd\_dist}\SpecialCharTok{\^{}}\DecValTok{2}
\NormalTok{v\_dist}
\end{Highlighting}
\end{Shaded}

\begin{verbatim}
## [1] 0.6174444
\end{verbatim}

\begin{Shaded}
\begin{Highlighting}[]
\CommentTok{\# variance}
\NormalTok{v\_theory }\OtherTok{\textless{}{-}}\NormalTok{ ((}\DecValTok{1}\SpecialCharTok{/}\NormalTok{l)}\SpecialCharTok{*}\NormalTok{(}\DecValTok{1}\SpecialCharTok{/}\FunctionTok{sqrt}\NormalTok{(n)))}\SpecialCharTok{\^{}}\DecValTok{2}
\NormalTok{v\_theory}
\end{Highlighting}
\end{Shaded}

\begin{verbatim}
## [1] 0.625
\end{verbatim}

\hypertarget{distribution}{%
\subsubsection{3 - Distribution}\label{distribution}}

Show that the distribution is approximately normal.

\begin{Shaded}
\begin{Highlighting}[]
\CommentTok{\# compare the distribution to a normal distribution}

\CommentTok{\# histogram for simulated means}
\FunctionTok{hist}\NormalTok{(means,}\AttributeTok{breaks=}\NormalTok{n,}\AttributeTok{prob=}\NormalTok{T,}\AttributeTok{col=}\StringTok{"pink"}\NormalTok{,}\AttributeTok{xlab =} \StringTok{"means"}\NormalTok{,}\AttributeTok{main=}\StringTok{"Distribution"}\NormalTok{,}\AttributeTok{ylab=}\StringTok{"density"}\NormalTok{)}

\CommentTok{\# overlay line for normal dist}
\NormalTok{x }\OtherTok{\textless{}{-}} \FunctionTok{seq}\NormalTok{(}\FunctionTok{min}\NormalTok{(means), }\FunctionTok{max}\NormalTok{(means), }\AttributeTok{length=}\DecValTok{100}\NormalTok{)}
\NormalTok{y }\OtherTok{\textless{}{-}} \FunctionTok{dnorm}\NormalTok{(x, }\AttributeTok{mean=}\DecValTok{1}\SpecialCharTok{/}\NormalTok{l, }\AttributeTok{sd=}\NormalTok{(}\DecValTok{1}\SpecialCharTok{/}\NormalTok{l}\SpecialCharTok{/}\FunctionTok{sqrt}\NormalTok{(n)))}
\FunctionTok{lines}\NormalTok{(x, y, }\AttributeTok{pch=}\DecValTok{22}\NormalTok{, }\AttributeTok{col=}\StringTok{"magenta"}\NormalTok{, }\AttributeTok{lty=}\DecValTok{5}\NormalTok{)}
\end{Highlighting}
\end{Shaded}

\includegraphics{StatisticalInferenceCourseProject_files/figure-latex/unnamed-chunk-8-1.pdf}

\end{document}
